\section{Codestatistik}
\subsection{Test Coverage}
Test Coverage wurde mit dem Tool \textbf{nose} durchgeführt. \\

\begin{table}[H]
\centering
    \begin{tabular}{|l|l|}
    \hline    
    \rowcolor{lightblue}
	Datei & Coverage [\%] \\ \hline
	src/base/Bbox.py & 85 \\ \hline    
	src/base/Node.py & 97 \\ \hline   
	src/base/Street.py & 92 \\ \hline  
	src/base/Tile.py & 98 \\ \hline   
	src/base/TileDrawer.py & 24 \\ \hline  
	src/data/MapquestApi.py & 100 \\ \hline   
	src/data/MultiLoader.py & 94 \\ \hline   
	src/data/StreetLoader.py & 100 \\ \hline   
	src/data/TileLoader.py & 100 \\ \hline   
	src/detection/BoxWalker.py & 100 \\ \hline   
	src/detection/NodeMerger.py & 89 \\ \hline  
	src/detection/StreetWalker.py & 100 \\ \hline   
	src/detection/deep/Convnet.py & 97 \\ \hline   
	src/detection/deep/training/Crosswalk\_dataset.py & 100 \\ \hline  
	src/detection/deep/training.py & 100 \\ \hline   
	src/role/Manager.py & 91 \\ \hline  
	src/role/WorkerFunctions.py & 70 \\ \hline
	\rowcolor{lightblue}
	Durchschnitt &   90.5 \\ \hline
    \end{tabular}
    \caption[Test Coverage]{Test Coverage}
\end{table}

\subsection{Codezeilen}
Die Codezeilen wurden mit Hilfe von \textbf{CLOC} \cite{CLOC} ausgezählt. \\

\begin{table}[H]
\centering
    \begin{tabular}{|p{3cm} |p{3cm} |p{3cm} |}
    \hline    
    \rowcolor{lightblue}
	Sprache & Dateien & Zeilen  \\ \hline   
	Python & 43 & 2045 \\ \hline
    \end{tabular}
    \caption[Codezeilen]{Codezeilen}
\end{table}

